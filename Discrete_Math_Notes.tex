\documentclass{article}
\usepackage{graphicx} % Required for inserting images
\usepackage{amsmath, amssymb, mathtools}

\title{Discrete Math Notes}
\author{Alex Feng}
\date{July 2023}

\begin{document}

\maketitle

\section{Set Theory}
\subsubsection*{Standard Symbols}
$\mathbb{P} = \{1,2,3,4,\dots \}$ \\
$\mathbb{N} = \{0,1,2,3,\dots \} \text{ (Natural Numbers, depends on definition)}$ \\
$\mathbb{Z} = \{ \dots, -3, -2, -1, 0, 1, 2, 3, \dots \}$ \\
$\mathbb{Q}$: the rational numbers \\
$\mathbb{R}$: the real numbers \\
$\mathbb{C}$: the complex numbers 

\subsubsection*{Definition: Finite Set}
A set is a finite set if it has a finite number of elements. Any set that isn't finite is an infinite set. 

\subsubsection*{Definition: Cardinality}
The number of different elements in a finite set $A$ is called its cardinality. The cardinality of $A$ is denoted by $|A|$. 

\subsubsection*{Definition: Subset}
Let $A$ and $B$ be sets. $A \subseteq B$ ($A$ is a subset of $B$) iff every element of $A$ is an element of $B$. (Additionally, iff $A \neq B$ and $A \subseteq B$, $A$ is a proper subset of $B$. $A$ is an improper subset of $A$) 

\subsubsection*{Definition: Set Equality}
Let $A$ and $B$ be sets. $A = B$ iff $A \subseteq B$ and $B \subseteq A$

\subsubsection*{Definition: Intersection}
For sets $A$ and $B$, $A \cap B = \{ x: x \in A \text{ and } x \in B \}$ intersection of $A$ and $B$.
\subsubsection*{Definition: Disjoint Sets}
Sets $A$ and $B$ are disjoint if $A \cap B = \emptyset$

\subsubsection*{Definition: Union}
The union of sets $A$ and $B$ is $A \cup B = \{ x: x \in A \text{ or } x \in B \}$

\subsubsection*{Definition: Universe}
The universal set, $U$, is the set of all elements under discussion for possible membership in a set. 

\subsubsection*{Definition: Complement of a set}
The complement of sets $A$ relative to $B$ is $B - A = \{ x: x \in B \text{ and } x \notin A \}$. The complement of $A$ is $A^c = U - A = \{ x \in U : x \notin A \}$

\subsubsection*{Definition: Symmetric Difference}
The symmetric difference of sets $A$ and $B$ is $A \oplus B = (A \cup B) - (A \cap B)$

\subsubsection*{Definition: Cartesian Products}
The Cartesian product of sets $A$ and $B$ is $A \times B = \{ (a,b) \mid a \in A \text{ and } b \in B \}$

\subsubsection*{Definition: Power Set}
The power set of any set $A$ is the set of all subsets of $A$, denoted $\mathcal{P} (A)$.

\subsubsection*{Definition: Generalized Set Operations}
For sets $A_1 , A_2 , \dots , A_n$
$$A_1 \cap A_2 \cap \cdots \cap A_n = \bigcap \limits_{i = 1}^{n} A_i$$
$$A_1 \cup A_2 \cup \cdots \cup A_n = \bigcup \limits_{i = 1}^{n} A_i$$
$$A_1 \times A_2 \times \cdots \times A_n = \bigtimes \limits_{i = 1}^{n} A_i$$
$$A_1 \oplus A_2 \oplus \cdots \oplus A_n = \bigoplus \limits_{i = 1}^{n} A_i$$

\section{Combinatorics} 
\subsubsection*{Theorem: Power Set Cardinality}
For a finite set $A$, $| \mathcal{P} (A) | = 2^{|A|}$

\subsubsection*{Definition: Permutation}
An ordered arrangement of $k$ elements selected from a set of $n$ elements, $0\leq k \leq n$, where no two elements of the arrangement are the same, is called a permutation of $n$ objects taken $k$ at a time. The total number of such permutations is denoted by $P(n,k)$

\subsubsection*{Theorem: Permutation Counting Formula}
$$P(n,k) = \prod_{j = 0} ^{k-1} (n-j) = \frac{n!}{(n-k)!}$$

\subsubsection*{Definition: Partition}
A partition of set $A$ is a set of one or more nonempty subsets of $A$, $A_1 , A_2 , A_3, \dots$, such that
$$A_1 \cup A_2 \cup A_3 \cup \cdots = A$$
$$\text{If } i \neq j \text{ then } A_1 \cap A_j = \emptyset$$

\subsubsection*{Theorem: The Basic Law of Addition}
If $\{ A_1 , A_2, \dots , A_n \}$ is the partition of a finite set $A$, then
$$|A| = \sum^n_{k=1} |A_k|$$

\subsubsection*{Theorem: Laws of Inclusion-Exclusion}
Given finite sets $A_1, A_2, A_3$, then
$$|A_1 \cup A_2| = |A_1| + |A_2| - |A_1 \cap A_2|$$
$$|A_1 \cup A_2 \cup A_3 | = |A_1| +|A_2| + |A_3| - (|A_1 \cap A_2 | + |A_1 \cap A_3 | + |A_2 \cap A_3|) + |A_1 \cap A_2 \cap A_3|$$

\subsubsection*{Definition: Binomial Coefficient}
Let $n$ and $k$ be nonnegative integers. The binomial coefficient $\binom{n}{k}$ represents the number of combinations of $n$ objects taken $k$ at a time, and is read ``$n$ choose $k$"

\subsubsection*{Theorem: Binomial Coefficient Formula}
If $n$ and $k$ are nonnegative integers with $0\leq k \leq n$, then the number $k$-element subsets of an $n$ element set is equal to 
$$\binom{n}{k} = \frac{n!}{(n-k)!k!}$$

\subsubsection*{Theorem: The Binomial Theorem}
If $n \geq 0$ and $x$ and $y$ are numbers, then
$$(x+y)^n = \sum^{n}_{k=0} \binom{n}{k} x^{n-k} y^k$$

\section{Logic}
\subsubsection*{Definition: Proposition}
A proposition is a sentence to which one and only one of the terms true or false can be meaningfully applied. 

\subsubsection*{Definition: Logical Conjunction}
If $p$ and $q$ are propositions, their conjunction, $p$ and $q$ (denoted $p \land q$), is defined by the truth table
$$\begin{array}{|c c|c|}
% |c c|c| means that there are three columns in the table and
% a vertical bar ’|’ will be printed on the left and right borders,
% and between the second and the third columns.
% The letter ’c’ means the value will be centered within the column,
% letter ’l’, left-aligned, and ’r’, right-aligned.
p & q & p \land q\\ % Use & to separate the columns
\hline % Put a horizontal line between the table header and the rest.
0 & 0 & 0\\
0 & 1 & 0\\
1 & 0 & 0\\
1 & 1 & 1\\
\end{array}$$

\subsubsection*{Definition: Logical Disjunction}
If $p$ and $q$ are propositions, their disjuction, $p$ or $q$ (denoted $p \lor q$), is defined by the truth table
$$\begin{array}{|c c|c|}
p & q & p \lor q \\
\hline
0 & 0 & 0\\
0 & 1 & 1\\
1 & 0 & 1\\
1 & 1 & 1\\
\end{array}$$

\subsubsection*{Definition: Logical Negation}
If $p$ is a proposition, its negation, not $p$, denoted $\neg p$, and is defined by the truth table
$$\begin{array}{c|c}
    p & \neg p \\
    \hline
    0 & 1\\
    1 & 0 \\
\end{array}$$


\subsubsection*{Definition: Conditional Statement}
The conditional statement ``If $p$ then $q$," denoted $p \to q$, is defined by the truth table
$$\begin{array}{|c c|c|}
p & q & p \to q \\
\hline
0 & 0 & 1\\
0 & 1 & 1\\
1 & 0 & 0\\
1 & 1 & 1\\
\end{array}$$

\subsubsection*{Definition: Converse}
the converse of the proposition $p \to q$ is the proposition $q \to p$

\subsubsection*{Definition: Contrapositive}
The contrapositive of the proposition $p \to q$ is the proposition $\neg q \to \neg p$

\subsubsection*{Definition: Logical Inverse}
The inverse of the proposition $p \to q$ is the proposition $\neg p \to \neg q$

\subsubsection*{Definition: Biconditional Proposition}
If $p$ and $q$ are propositions, the biconditional statement ``$p$ if and only if $q$," denoted $p \leftrightarrow q$, is defined by the truth table
$$\begin{array}{|c c|c|}
p & q & p \leftrightarrow q \\
\hline
0 & 0 & 1\\
0 & 1 & 0\\
1 & 0 & 0\\
1 & 1 & 1\\
\end{array}$$

\subsubsection*{Definition: Proposition Generated by a Set}
Let $S$ be any set of propositions. A proposition generated by $S$ is any valid combination of propositions in $S$ with conjunction, disjunction, and negation. More precisely, 

a. If $p \in S$, then $p$ is a proposition generated by $S$

b. If $x$ and $y$ are propositions generated by $S$, then so are $$(x), \neg x, x \lor y,\text{ and }x \land y$$

\subsubsection*{Definition: Tautology}
An expression involving logical variables that is true in all cases is a tautology. The number $1$ is used to symbolize a tautology.

\subsubsection*{Definition: Contradiction}
An expression involving logical variables that is false for all cases is called a contradiction. The number $0$ is used to symbolize a contradiction. 

\subsubsection*{Definition: Equivalence}
Let $S$ be a set of propositions and let $r$ and $s$ be propositions generated by $S$. $r$ and $s$ are equivalent iff $r \leftrightarrow s$ is a tautology. The equivalence of $r$ and $s$ is denoted $r \iff s$.

\subsubsection*{Definition: Implication}
Let $S$ be a set of propositions and let $r$ and $s$ be propositions generated by $S$. We say that $r$ implies $s$ if $r \rightarrow s$ is a tautology. We write $r \implies s$ to indicate this implication. 

\subsubsection*{Definition: The Sheffer Stroke}
The Sheffer Stroke is the logical operator defined by the following truth table: 
$$\begin{array}{|c c|c|}
p & q & p \mid q \\
\hline
0 & 0 & 1\\
0 & 1 & 1\\
1 & 0 & 1\\
1 & 1 & 0\\
\end{array}$$

\subsubsection*{Definition: Mathematical System}
A mathematical system consists of:

1. A set or universe, $U$.

2. Definitions: sentences that explain the meaning of concepts that relate to the universe. Any term used in describing the universe itself is said to be undefined. All definitions are given in terms of these undefined concepts of objects.

3. Axioms: assertions about the properties of the universe and rules for creating and justifying more assertions. these rules always include the system of logic that we have developed to this point. 

4. Theorems: the additional assertions mentioned above. 

\subsubsection*{Definition: Theorem}
A true proposition derived from the axioms of a mathematical system is called a theorem.

\subsubsection*{Definition: Proof}
A proof of a theorem is a finite sequence of logically valid steps that demonstrate that the premises of a theorem imply its conclusion. 

\subsubsection*{Definition: Proposition over a Universe}
Let $U$ be a nonempty set. A proposition over $U$ is a sentence that contains a variable that can take on any value in $U$ and that has a definite truth value as a result of any such substitution.

\subsubsection*{Definition: Truth Set}
If $p$ is a proposition over $U$, the truth set of $p$ is $T_p = \{ a \in U \mid p(a) \text{ is true } \}$.

\subsubsection*{Definition: Tautologies and Contradictions over a Universe}
A proposition over $U$ is a tautology if its truth set is $U$. It is a contradiction if its truth set is empty. 

\subsubsection*{Definition: Equivalence of propositions over a universe}
For two propositions $p$ and $q$, if $p \iff q$ then $T_p = T_q$

\subsubsection*{Definition: Implication for propositions over a universe}
If $p$ and $q$ are propositions over $U$, $p \implies q$ if $p \rightarrow q$

\subsubsection*{Theorem: The Principle of Mathematical Induction}
Let $p(n)$ be a proposition over the positive integers. If

1. $p(1)$ is true, and

2. for all $n \geq 1$, $p(n) \implies p(n+1)$

then $p(n)$ is a tautology.

\subsubsection*{Definition: The Existential Quantifier}
If $p(n)$ is a proposition over $U$ with $T_p \neq \emptyset$, we abbreviate ``There exists an $n$ in $U$ such that $p(n)$ (is true)" using $(\exists n )(p(n))$

\subsubsection*{Definition: The Universal Quantifier}
``For all $n$ in $U$, $p(n)$ (a proposition over $U$ with $T_p = U$)" can be abbreviated using $(\forall n)(p(n))$

\section{Relations}
\subsubsection*{Definition: Relation}
A relation from sets $A$ into $B$ is any subset of $A \times B$.

\subsubsection*{Definition: Relation on a Set}
A relation from a set $A$ into itself is called a relation on $A$. 

\subsubsection*{Definition: Divides}
$a, b \in \mathbb Z, a \neq 0$. $a$ divides $b$, $a \mid b \iff \exists k \in \mathbb Z , a k = b$

\subsubsection*{Definition: Composition of Relations}
Let $r$ be a relation from a set $A$ into a set $B$, and let $s$ be a relation from $B$ into a set $C$. The composition of $r$ with $s$, written $rs$, is the set of pairs of the form $(a,c) \in A \times C$, where $(a,c) \in rs \iff \exists b \in B, (a,b)\in r \land (b,c) \in s$.

\subsubsection*{Definition: Reflexive Relation}
Let $A$ be a set and let $r$ be a relation on $A$. Then $r$ is reflexive iff $\forall a \in A, ara$.

\subsubsection*{Definition: Antisymmetric Relation}
Let $A$ be a set and let $r$ be a relation on $A$. Then $r$ is antisymmetric iff $arb \land a \neq b \implies \neg bra$.

\subsubsection*{Definition: Transitive Relation}
Let $A$ be a set and let $r$ be a relation on $A$. $r$ is transitive iff $arb \land brc \implies arc$

\subsubsection*{Definition: Partial Ordering}
A relation on a set $A$ that is reflexive, antisymmetric, and transitive is called a partial ordering on $A$. A set on which there is a partial ordering relation defined is called a partially ordered set or poset. 

\subsubsection*{Definition: Symmetric Relation}
let $r$ be a relation on a set $A$. $r$ is symmetric iff $arb \implies bra$.

\subsubsection*{Definition: Equivalence Relation}
A relation $r$ on a set $A$ is called an equivalence relation iff it is reflexive, symmetric, and transitive. 

\subsubsection*{Definition: Equivalence Classes}
Let $r$ be an equivalence relation on $A$, and $a \in A$. The equivalence class of $a$ is the set, $\left[ a \right]$, of all elements to which $a$ is related.
$$\left[ a \right] = \{ b \in A : arb \}$$
The set of all equivalence classes with respect to $r$ is denoted $A / r$, read ``$A$ mod $r$."

\subsubsection*{Theorem: Equivalence Class Partition}
Let $r$ be an equivalence relation on $A$. Then the set of all distinct equivalence classes determined by $r$ form a partition of $A$ denoted $A / r$ and read ``$A$ mod $r$."

\subsubsection*{Definition: Congruence Modulo $n$}
Let $n \in Z ^+, n \geq 2$. We define congruence modulo $n$ to be the relation $\equiv_n$ defined on the integers by 
$$a \equiv_n b \iff n \mid (a -b)$$

\section{Functions}
\subsubsection*{Definition: Function}
A function from a set $A$ into a set $B$ is a relation from $A$ into $B$ such that each element of $A$ is related to exactly one element of the set $B$. The set $A$ is called the domain of the function and the set $B$ is called the codomain. 

\subsubsection*{Definition: The Set of Functions Between Two Sets}
Given two sets, $A$ and $B$, the set of all functions from $A$ into $B$ is denoted $B^A$. 

\subsubsection*{Definition: Image of an element under a function}
Let $f : A \rightarrow B$, read ``Let $f$ be a function from the set $A$ into the set $B$." If $a \in A$, then $f(a)$ is used to denote that element of $B$ to which $a$ is related. $f(a)$ is called the image of $a$, or, more precisely, the image of $a$ under $f$. We write $f(a) = b$ to indicate that the image of $a$ is under $b$. 

\subsubsection*{Definition: Range of a Function}
If $f : X \rightarrow Y$, then the range of $f$, is
$$f(X) = \{ f(a) \mid a \in X \} = \{b \in Y \mid \exists a \in X , f(a) = b \}$$

\subsubsection*{Definition: Injection}
A function $f : A \rightarrow B$ is injective (one-to-one) if
$$\forall a, b \in A, a \neq b \implies f(a) \neq f(b)$$

\subsubsection*{Definition: Surjection}
A function $f: A\rightarrow B$ is surjective (onto) if
$$f(A) = B$$
which is equivalent to
$$\forall b \in B, \exists a \in A , f(a) = b$$

\subsubsection*{Definition: Bijection}
A function $f: A \rightarrow B$ is bijective (one-to-one and onto) if it is both injective and surjective. 

\subsubsection*{Definition: Cardinality}
Two sets are said to have the same cardinality if there exists a bijection between them. If a set has the same cardinality as the set $\{1, 2, 3, \dots n \}$ , then we say its cardinality is $n$. 

\subsubsection*{Definition: Countable Set}
If a set is finite or has the same cardinality as the set of positive integers, it is called a countable set. 

\subsubsection*{Theorem: The Pigeonhole Principle}
let $f$ be a function from a finite set $X$ into a finite set $Y$. If $n \geq 1$ and $|X| > n |Y|$, then there exists an element of $Y$ that is the image under $f$ of at least $n+1$ elements of $X$. 

\subsubsection*{Definition: Equality of Functions}
let $f, g : A \rightarrow B$. $$f = g \iff \forall x \in A, f(x) = g(x)$$

\subsubsection*{Definition: Composition of Functions}
let $f: A \rightarrow B$ and $g: B \rightarrow C$. The composition of $f$ followed by $g$, written $g \circ f$, is a function from $A$ into $C$ defined by $(g \circ f)(x) = g(f(x))$, which is read ``$g$ of $f$ of $x$."

\subsubsection*{Theorem: Function Composition Associativity}
If $f: A \rightarrow B, g: B \rightarrow C, \text{ and } h: C \rightarrow D$, then $h \circ ( g \circ f) = (h \circ g) \circ f$.

\subsubsection*{Definition: Powers of Functions}
Let $f: A \rightarrow A$. 
$$f^1 = f, \text{ that is, }\forall a \in A, f^1 (a) = f(a)$$
$$\text{For } n \geq 1, f^{n+1} = f \circ f^n$$

\subsubsection*{Theorem: Composition of Injections}
If $f: A \rightarrow B$ and $g: B \rightarrow C$ are injections, then $g \circ f : A \rightarrow C$ is an injection.

\subsubsection*{Theorem: Composition of Surjections}
If $f: A \rightarrow B$ and $g: B \rightarrow C$ are surjections, then $g \circ f: A \rightarrow C$ is a surjection. 

\subsubsection*{Definition: Identity Function}
For any set $A$, the identity function on $A$ is a function from $A$ onto $A$, denoted $i$ (or, more specifically, $i_A$) such that $\forall a \in A, i(a) = a$ 

\subsubsection*{Definition: Inverse of a Function on a Set}
Let $f: A \rightarrow A$. If there exists a function $g: A \rightarrow A$ such that $g \circ f = f \circ g = i$, then $g$ is called the inverse of $f$ and denoted by $f^{-1}$, read ``$f$ inverse."

\subsubsection*{Theorem: Bijection Inverse}
Let $f: A \rightarrow A$. $f^{-1}$ exists iff $f$ is a bijection. 

\subsubsection*{Definition: Permutation}
A bijection of a set $A$ into itself is called a permutation of $A$. 

\subsubsection*{Definition: Inverse of a Function (General Case)}
Let $f: A \rightarrow B$. If there exists a function $g: B \rightarrow A$ such that $g \circ f = i_A$ and $f \circ g = i_B$, then $g$ is called the inverse of $f$ and is denoted by $f^{-1}$, read ``$f$ inverse."

\subsubsection*{Theorem: Existence of Inverse}
Let $f: A \rightarrow B$. $f^{-1}$ exists iff $f$ is a bijection. 
\end{document}
