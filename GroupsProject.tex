\documentclass{article}
\usepackage{graphicx} % Required for inserting images
\usepackage{amsmath}
\usepackage{tikz}
\usetikzlibrary{positioning, graphs}
\title{The Quaternion Group and Klein Four Group}
\author{Alexander Feng}
\date{November 2023}

\begin{document}

\maketitle

\section{The Quaternion Group}
The Quaternion group, $Q_8 = \{\pm 1 , \pm i, \pm j, \pm k \}$, is a finite group of order 8. It is not abelian, and its binary operation is as follows: For all $a \in Q_8$,
$$1 \cdot a = a \cdot 1 =a$$
$$(-1) \cdot (-1) = 1$$
$$(-1 ) \cdot a = a, a\cdot (-1) = -a$$
$$i \cdot i = j \cdot j = k \cdot k = -1 $$
\begin{align*}
i \cdot j &= k, & j \cdot i &= -k \\
j \cdot k &= i, & k \cdot j &= -i \\
k \cdot i &= j, & i \cdot k &= -j.
\end{align*}
The order of 1 is 1, and the order of $-1$ is 2. The order of the rest of the elements is 4. $Q_8$ can be generated by any two elements other than 1 and $-1$, for example, $Q_8 = \langle -i, k \rangle$. All of its subgroups are normal. 

\begin{center}
\begin{tikzpicture}
  % Nodes
  \node (1) at (0,4) {$Q_8$};
  \node (i) at (-2,2) {$\langle -i \rangle = \langle i\rangle$};
  \node (j) at (0,2) {$\langle -j \rangle = \langle j\rangle$};
  \node (k) at (2,2) {$\langle -k \rangle = \langle  k\rangle $};
  \node (neg1) at (0,0) {$\langle -1 \rangle $};
  \node (Q8) at (0,-2) {$\langle 1 \rangle$};

  % Edges
  \draw (1) -- (i);
  \draw (1) -- (j);
  \draw (1) -- (k);
  \draw (i) -- (neg1);
  \draw (j) -- (neg1);
  \draw (k) -- (neg1);
  \draw (neg1) -- (Q8);
\end{tikzpicture}
\end{center}

\subsection*{Applications}
Quaternions in the form $q = w + xi + yj + zk$ where $i, j,k$ multiplication follows the rules above can be used to represent rotations in 3D space. Let 
$$\vec u = (u_x , u_y , y_z)$$
be a unit vector that represents the axis of rotation. Let 
$$q = \sin \left(\frac{\theta}{2} \right) + \cos \left(\frac{\theta}{2} \right) (u_x i + u_y j + u_z k) $$
where $\theta$ is the rotation angle, and represent a vector $v$ as 
$$v = v_x i + v_y j + v_z k$$
The vector $v$ rotated an angle of $\theta$ around axis $\vec u$ is
$$v' = q \cdot v \cdot q^{-1}$$
Where $q^{-1}$ is the inverse of $q$, and can easily be calculated (since it is a unit vector) by
$$q^{-1} = \sin \left(\frac{\theta}{2} \right) - \cos \left(\frac{\theta}{2} \right) (u_x i + u_y j + u_z k) $$
\section{The Klein Four Group}
The Klein Four Group (Vierergruppe), $V_4 = \{1, a, b, c\}$, is a finite abelian group of order 4. Its binary operation is as follows: For all $x \in V_4$,
$$1x = x1 = x$$
$$x^2 = 1$$
$$ab = c$$
$$bc = a$$
$$ac = b$$
The order of 1 is 1, and the order of the other elements is 2. $V_4$ can be generated by any two elements from $\{ a, b, c \}$. All of its subgroups are normal. Observe that $Q_8 / \langle -1 \rangle \cong V_4$.
\begin{center}
\begin{tikzpicture}
  % Nodes
  \node (V4) at (0,4) {$V_4$};
  \node (ea) at (-2,2) {$\langle a\rangle$};
  \node (eb) at (0,2) {$\langle b\rangle$};
  \node (ec) at (2,2) {$\langle c\rangle$};
  \node (e) at (0,0) {$\langle 1\rangle$};

  % Edges
  \draw (V4) -- (ea);
  \draw (V4) -- (eb);
  \draw (V4) -- (ec);
  \draw (ea) -- (e);
  \draw (eb) -- (e);
  \draw (ec) -- (e);
\end{tikzpicture}
\end{center}

\subsection*{Applications}
The Klein Four Group is used in coding theory. In particular, it is used for error detection/correction, digital data transmission, and data storage. 
\end{document}
